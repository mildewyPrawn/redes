\documentclass[spanish,12pt,letterpaper]{article}

\usepackage[english]{babel}
\usepackage[table]{xcolor}
\usepackage[utf8]{inputenc}
\usepackage{amsmath}
\usepackage{amssymb}
\usepackage{authblk}
\usepackage{csquotes}
\usepackage{enumerate}
\usepackage{float}
\usepackage{forest}
\usepackage{geometry}
\usepackage{graphicx}
\usepackage{listings}
\usepackage{xcolor}

\geometry{
  a4paper,
  total={170mm,257mm},
  left=20mm,
  top=20mm,
}

\definecolor{codegreen}{rgb}{0,0.6,0}
\definecolor{codegray}{rgb}{0.5,0.5,0.5}
\definecolor{codepurple}{rgb}{0.58,0,0.82}
\definecolor{backcolour}{rgb}{0.95,0.95,0.92}

\lstdefinestyle{mystyle}{
  backgroundcolor=\color{backcolour},
  commentstyle=\color{codegreen},
  keywordstyle=\color{orange},
  numberstyle=\tiny\color{codegray},
  stringstyle=\color{codepurple},
  basicstyle=\ttfamily\footnotesize,
  breakatwhitespace=false,
  breaklines=true,
  captionpos=b,
  keepspaces=true,
  numbers=right,
  numbersep=5pt,
  showspaces=false,
  showstringspaces=false,
  showtabs=false,
  tabsize=2
}

\lstset{style=mystyle}

\title{Examen Redes de Computadoras}
\author{Ángel Iván Gladín García\\
  Emiliano Galeana Araujo}
\affil{Facultad de ciencias, UNAM}
\date{Fecha de entrega: Lunes 30 de marzo de 2020}

\begin{document}

\maketitle

\section{Parte teórica}

\section{Parte práctica}
\subsection{Considerar una red de la siguiente forma}
Llevar a cabo lo siguiente:
\begin{itemize}
\item Asignar direcciones de un \textit{segmento privado de clase B} a los
  equipos de la Red A.
\item Establecer la dirección del \textit{gateway} en la Red A.
\item Indica la configuración de red que se debe usar en los equipos de la Red A
  (\textit{dirección de red, gateway, máscara de subred} y otros que consideres
  necesarios).
\item Hacer uso de un \textit{segmento de red público de clase C} en los equipos
  de la Red B.
\item Establecer la dirección del \textit{gateway} en la Red B.
\item Indica la configuración de red que se debe usar en el servidor de la Red B
  (\textit{dirección de red, gateway, máscara de subred} y otros que consideres
  necesarios).
\item Asignar una \textit{dirección de red} a las interfaces correspondientes del
  \textit{router}, de acuerdo a las redes conectadas.
\item Incluir en el diagrama un \textit{DNS} en una red independiente, haciendo
  uso de un segmento de \textit{red de clase C}, con direcciones públicas.
\item Considerando que todos los clientes y servidores involucrados son equipos
  Linux, indica la forma en la que se debe establecer la configuración de
  \textit{red estática} y la configuración necesaria para llevar a cabo la
  \textit{resolución de nombres} en el sistema (archivos y comandos necesarios).
\item Por cuestiones de simplicidad, asignar a cada dispositivo  una
  \textit{dirección física de 4 números hexadecimales}.
\item Explicar con detalle el procedimiento para llevar a cabo la coneción entre
  una computadora de la Red A al servidor de la Red B, considerando lo siguiente:
  \begin{itemize}
  \item Se debe establecer el nombre del servidor con base en tu nombre y
    apellidos.
    
    jose-luis.torres.local o juan.camacho.local
  \item El servidor de la Red B cuenta con una aplicación de "servidor Web" el
    cual solamente acepta peticiones mediante \textit{HTTP} version 1.1 en el
    puerto 54321.
  \item El cliente establecerá una conexión para obtener el documento /paginas
    /directorio.html.
  \item El cliente debe hacer uso de un puerto, fuera del rango establecido para
    los "puertos bien conocidos" y "puertos registrados", para llevar a cabo la
    conexión.
  \item El \textit{gateway} de la Red A implementa \textit{NAT}, haciendo uso de
    una dirección de un segmento de \textit{clase C}.
  \item Se debe incicar detalladamente los passos que se siguen en el cliente de
    la Red A que hacen uso de una conexión con un \textit{medio guiado}, para
    establecer la conexión con el servidor, incluyendo los \textit{bloques de
      datos} que se general al pasar por cada una de las capas de TCP/IP. Se debe
    mostrar la forma en la que se aplica el ``encapsulamiento'' (En las clases se
    presentaron ejemplos de esto)
  \item En la \textit{Capa de Enlace} se debe considerar el uso de ``relleno de
    bits'' para llevar a cabo el \textit{envío de tramas}.
  \item Se debe mostrar la forma en la que se ``desencapsulan'' los paquetes al
    llegar al \textit{router} y la forma en la que se vuelven ``encapsular'' para
    su reenvío.
  \end{itemize}
\end{itemize}

\subsection{Explicar cuáles son las características que debe tener un código para
  permitir llevar a cabo \textit{detección} o \textit{corrección} de errores}

En este contexto, responde las siguientes preguntas:
\begin{itemize}
\item ¿Qué características debe tener un código para permitir la detección de
  errores de un máximo de n bits?
\item ¿Qué características debe tener un código para permitir la corrección de
  errores de un máximo de n bits?
\item ¿Cuál es el número mínimo de mensajes que deben enviar las dos partes para
  garantizar una comunicación libre de errores?
\end{itemize}

\subsection{¿Cuáles son las tres características que componen al World Wide Web?}

\begin{itemize}
\item URI (Uniform Resource Identifier), que es un sustema universal para
  referenciar fuentes en la Web, así como páginas Web.
  
\item HTTP (HyperText Transfer Protocol) especifica cómo el navegador y el
  servidor se comunican entre si.
\item HTML (HyperText Markup Language) se usa para definir la estructura y el
  contenido de los documentos con hipertexto.
\end{itemize}


\subsection{Explica el funcionamiento de CRC}

Comprobación de Redundancia Cíclica o Control de Redundancia Cíclica, del inglés
(Cyclic Redundancy Check), es una función diseñada para detectar cambios
accidentales en datos de computadoras y es comunmente usada en redes y
dispositivos de almacenamiento.

\subsubsection{Funcionamiento}
A cada bloque de datos le corresponde una secuencia fija de números binarios
conocida como código CRC (Se calcula con una misma función para cada bloque).
Ambos se envían o almacenan juntos. Cuando un bloque de datos es leído o recibido
, dicha función es aplicada nuevamente al bloque, si el código CRC generado no
coincide con el código CRC original, entonces significa que el bloque contiene
un error. Eso hará que el dispositivo intente solucionar el error releyendo el
bloque o requiriendo que sea enviado nuevamente. Si ambos códigos coinciden,
entonces se asume que el bloque no contiene errore (Existe una remota posibilidad
de que exista un error sin detectar).

\subsubsection{Ejercicio}

Considerar el siguiente polinomio:

CRC-16 = 11000000000000101 = $X^{16} + X^{15} + X^{2}+ 1$

1101011011

Calcula el resultado de aplicar este polinomio a la cadena de texto ``Hola
mundo'', mediante CRC. Incluye el procedimiento para obtener el resultado.

``Hola mundo'' = \texttt{01001000 01101111 01101100 01100001 00100000 01101101
  01110101 01101110 01100100 01101111}

Agregamos los bits redundantes... Como tenemos un polinomio de 17 bits, agregamos
16 bits.

\texttt{01001000 01101111 01101100 01100001 00100000 01101101 01110101 01101110
  01100100 01101111 00000000 00000000}

Procedemos a hacer la división aydándonos con un \textit{XOR}. No pondremos toda
la división aquí, pero adjuntaremos una hoja de cálculo donde se puede revisar,
sin embargo, vamos a notar los casos que creemos importantes.


\subsection{Extra}
\begin{itemize}
\item Describe las potenciales \textit{debilidades} del protocolo \textit{DNS},
  ¿Qué mecanismos de seguridad podrías implementar sobre este protocolo para
  mitigar dichas vulnerabilidades?

  Nos centramos en el concepto de que el \textit{DNS} funciona como un directorio
  telefónico, donde nosotros ponemos el nombre de dominio y nos regresa la
  dirección IP.

  La debilidad que creemos podría ser más importante, es que el protocolo al
  pedir por un nombre de dominio, nos regrese la dirección de un sitio malicioso
  o una dirección ajena a la que pedimos. Esto es importante, pues no se podrían
  resolver las consultas que se hagan. Y la manera que se nos ocurrió para
  mitigar esta vulnerabilidad sería que el protocolo pueda comparar el resultado
  con resultados anteriores o resultados de computadoras vecinas en nuestra red.
  De esta manera si algún protocolo regresa una dirección distinta a la de los
  demás, podríamos no tomarla en cuenta, e ir por la respuesta que la mayoría de
  computadoras efectuando el protocolo regresaron.

  
\item Explicar la diferencia entre ``codificar'' y ``cifrar''. Incluir ejemplos
  de cada uno de ellos.

  Cuando hablamos de 'cifrar', nos referimos a ocultar la información basándonos
  en la sintáxis del mensaje. Podemos utilizar algoritmos para realizar esto, los
  cuales suelen utilizar una clave para transformar la estructura del mensaje.

  Si hablamos de 'codificar' se basa en alterar la semántica del mensaje, lo que
  está relacionado con el significado del mensaje.

  Ejemplos que podemos aplicar a lo anterior, podemos \textit{cifrar} mensajes en
  aplicaciones de mensajería (Telegram, WhatsApp, etc). Y un ejemplo de \textit{
    codificar} puede ser en la computadora con emojis, cuando representamos el
  emoji con caracteres, por ejemplo \texttt{U+1F601}, y podemos verlo en la
  terminal como un emoji.
\end{itemize}

\begin{thebibliography}{9}
\bibitem{lamport94}
  https://www.varonis.com/blog/what-is-dns/

  https://www.welivesecurity.com/la-es/2016/12/07/codificacion-o-cifrado-diferencia/?fbclid=IwAR3R2RuHwb9grV7cKTuIXJOhtLbkw2AblOub1RrrMujh5nKns0tuqTv8fYE

  https://www.cs.mcgill.ca/~rwest/wikispeedia/wpcd/wp/w/World_Wide_Web.htm

  https://www.ecured.cu/Comprobaci%C3%B3n_de_redundancia_c%C3%ADclica


\end{thebibliography}



\end{document}
